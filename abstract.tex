\begin{abstract}
While the synthesis of gene-sized DNA fragments is becoming a commodity, their assembly into large plasmids or artificial chromosomes remains cost and time-intensive. Such constructs can be designed using specialized software, and built on robotic systems, but few tools support the intermediary planning step, where researchers must devise a manufacturing strategy adapted to their DNA procurement and assembly capabilities. DNA Weaver aims at providing a generic framework to automate this step, where researchers first design a supply network representing their options in terms of cloning methods (Golden Gate assembly, Gibson assembly, PCR extraction of sequences from existing templates), DNA vendors, and available repositories of reusable DNA. Given the sequence to manufacture, the framework then uses efficient graph algorithms to find an optimal (or fast but sub-optimal) suitable assembly strategy, and generates comprehensive assembly planning reports. Here, we show that the framework can address various DNA construction scenarios, from site-directed mutagenesis to multi-step artificial chromosome assembly, and  can model many limitations of different cloning methods, making it a versatile and tool to support the transition from design to manucturing in biofoundries and synthetic biology laboratories. 

\end{abstract}
