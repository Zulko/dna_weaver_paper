
and DNA procurement of finding a fitting manufacturing strategy often rely on human judgment. DNA sequences to be designed via softwareGene-length synthetic DNA is becoming a commodity, but its assembly into larger constructs, such as plasmids and chromosomes, remains cost- and time-intensive. Automation solutions have already greatly simplified the task, allowing hundreds of DNA sequences to be designed via software and assembled on robotic systems. However the intermediary steps, including the selection of fitting cloning strat for Researchers increasingly rely on software to design construct sequences, and robotic systems for at once While software can assist with the design of , and robotic systems have considerably sped up cloning , laying out the details of manufacturing is often a bottleneck, as it relies Or the available cloning expertise in a researcher's group. Software tools can help with sequence design, but the selection of an adequate manufacturing strategy relies on expert judgment and is determined by the choice of DNA providers, assembly method.
While robotic systems enable DNA synthesis and testing at unprecedented rates, curbing project costs and complexity requires thorough operational planning. Software can assist with DNA sequence design, but few tools automate the selection of befitting manufacturing str


 and assembled on robotic systems. However the intermediary steps, including the selection of fitting cloning strategies and DNA suppliers, and the design of multi-step assembly plans, are often left to the researcher's judgment.
DNA weaver allows researchers to model available resources and devise optimal strategies one-step or hierarchical assemblies, comparing providers, assembly techniques, and parts reuse. network

ategies for a given sequence. In particular, the assembly of large plasmids or chromosomes from smaller DNA fragments

While robotic systems enable DNA synthesis and testing at unprecedented rates, curbing project costs and complexity requires thorough operational planning. Software can assist with DNA sequence design, but few tools automate the selection of befitting manufacturing strategies for a given sequence. In particular, the assembly of large plasmids or chromosomes from smaller DNA fragments generally rel

generally relies on expert judgment rather than software. DNA Weaver aims at proposing we propose a framework aiming at modelling knowledge . The framework can model commercial DNA sources (such as commercial providers DNA librairies) and assembly techniques (such as Gibson Assembly or Golden Gate assembly). Users define a supply network defining the relation between, and efficient graph algorithms quickly give optimal or approximate solutions to  cannot be automated at scale. We present DNA Weaver, a software framework which aims at representing .enabling users to model . the   becomes critical large project planning and large molecular biology projects has become a bottleneck. construction and testing, the choice of strategy to build DNA manufacturing and screening,
  the design of strategies to build DNA is often carried out by humaof DNA constructs is enjoying
  troughput increase under lab automation, the devising of a strategy the DNA constructions are often 
  
 
DNA weaver allows researchers to model available resources and devise optimal strategies one-step or hierarchical assemblies, comparing providers, assembly techniques, and parts reuse. network