\section{Introduction}
The development of new Synthetic Biology applications often require to iterate through cycles of genetic sequence design, manufacturing, and testing. The duration of a cycle (typically weeks or months) has been considerably reduced by the advance of automation in recent years. On one hand, progress in genetic computer-aided design have made it possible to automatically generate sequence designs to tune promoter activity \citep{Brown2017, Gilman2019}, or build genetic circuits with predictible logics \citep{Nielsen2016}. Meanwhile, increasingly affordable robotic systems can perform plasmid assembly and verification operations at a rate of hundreds of plasmids a week \citep{Storch2019, Shapland2015}. However, the transition between design and manufacturing, consisting in determining the molecular operations necessary to obtain a given genetic sequence, can be a bottleneck of the development cycle, notably in challenging scenarios where the input human experts is required.

Sequences small enough to be assembled from oligonucleotides can be ordered from one of many existing commercial companies, at ever more affordable prices \citep{Kosuri2014}. However, the choice of a commercial provider is difficult to automate due to differences in pricing, lead times, and sequence constraints, such as bounds on the accepted sequence lengths, or tolerance to the presence of synthesis-impeding patterns, as described in \citep{Oberortner2017}. Alternatively, gene-sized sequences can also be obtained at lower costs from physical parts repositories \citep{Vilanova2014, Guo2015} or extracted from an existing construct or genome via PCR \citep{Guo2015,Timmons2020}.

Larger synthetic constructs, such as genetic circuits or artificial chromosomes, must be assembled from smaller fragments. This can be done using a variety of techniques, each presenting advantages and limitations. For instance, methods using type-2s restriction enzymes, such as Golden Gate assembly \citep{Engler2008}, allow the routine assembly of up to 20 fragments at once \citep{Martella2017, Potapov2018}, and standards based on this approach allow researchers to efficiently re-use standardized genetic parts between projects \citep{Iverson2016, Andreou2017}. However, such methods cannot be used to build sequences containing restriction sites of the type-2s enzyme used. Moreover, the method requires the careful design of short, 4bp assembly overhangs \citep{Potapov2018}, and the order of the standardized parts in the final construct cannot be changed without first modifying the fragment overhangs via PCR or re-synthesis. In contrast, the Gibson Assembly \citep{Gibson2011} and Ligase Cycling Reaction (LCR) Assembly \citep{Kok2014} methods are not impacted by internal restriction sites, and LCR Assembly allows to change the assembly order of DNA fragments via simple oligonucleotide designs, but both methods are sensitive to the secondary structure of the fragment sequences, and by homologies between different regions of the sequence to assemble.
As a result of the respective limitations of each technique, there is no prevailing strategy suitable to every scenario, and large assembly projects may use a combination of techniques, left to the decision of the researcher.

Projects involving large numbers of assembly operations via pre-selected assembly techniques are often supported by specialized, custom-built software. A first category of projects focuses on the assembly of artificial chromosomes \textit{from scratch} (i.e. without reusing DNA from an existing template). In the Sc2.0 projects \citep{Richardson2017}, 75bp oligonucleotides are first assembled into 750bp "building blocks". The blocks are then assembled into 3kb "minichunks", then 10kb chunks, and finally 30kb megachunks, which are inserted in the chomosome. Genome Partitioner \citep{Christen2017} uses different intermediary constructs sizes (1kb, 4kb, 20kb, and chromosome insertion), and the OGABE assembly method \citep{Tsuge2015} enables the construction of a viral chromosome in a single assembly of fifty 1kb fragments. The software supporting the manufacturing of these projects according to their respective protocols uses a top-down approach, in which the target sequence is split into overlapping blocks, which may in turn be split into smaller blocks, etc.

Another category of projects focuses on the assembly of genetic circuits using standard genetic parts. Part-based assembly standards may be published together with companion software to help users devise assembly protocols. Examples of such specialized software include Golden Braid 3.0 for Golden Braid Assembly \citep{Vazquez-Vilar2017}, and Loop Designer for Loop Assembly \citep{Pollak2019}. While these frameworks are tied to specific assembly methods, other software tools achieve more versatility by devising assembly plans to build any combination of reused parts, by repurposing the parts via PCR. The j5  framework \citep{Hillson2012} notably devises parts standardization for assembly with a wide choice of techniques, including Gibson Assembly and Golden Gate Assembly, from user-provided parts and constructs designs. the framework can also determine which of commercial ordering of PCR-extension is the best way of obtaining a DNA fragment. The RavenCAD framework \citep{Appleton2014} uses dynamic programming to compute hierarchical assembly plans from base parts to a desired construct, including PCR extension of the parts to allow their assembly via the BioBrick or Golden Gate method.

Many common scenarios in DNA construction mix reused DNA and fragments built from scratch. Figure 1C illustrates the scenario where a . Large genes, of the order of 10kb, may need to be ordered in 

REPP, which enables to build a plasmid from parts reused from a repository or commercially ordered \citep{Timmons2020}.


% The main difficulty is that DNA construction projects can employ various strategies, making more or less use of pre-existing constructs or de-no synthesized DNA.
% On one end of the spectrum, a construct can be entirely assembled from pre-existing genetic parts.
% Parts repositories such as Biobrick or YeastFab provide standardized parts which can be assembled hierarchically to create genetic circuits (Figure 1A), generally via restriction ligation.
% In particular, Golden Gate-based assembly kits (MoClo, Golden Braid), are generally assembled via type-2S restriction ligation. res standards rely on.


% On the other end of the spectrum, projects have focused on the assembly 

% such as Biobrickstandard genetic parts can be assembled from standard parts, for instance BioBricks (REF) can be done from


% Many projects come in-between.



