\section{Discussion}

We have presented a new framework allowing the intuitive definition of DNA construction problems via supply networks, and relying on efficient sequence decomposition algorithms to find optimal assembly plans for arbitrary sequences. The usefulness of such a framework depends on its ability to capture the subtleties of the different DNA construction processes. We have shown that many practical constraints of DNA assembly projects could be modelled either at the level of the supply network, or by amending the sequence decomposition graphs. As a result, the framework can provide realistic estimations of a project's cost and complexity, in order to support research groups or biofoundries in their planning and resource allocation. But could the framework fully automate the transition from design to robotic manufacturing, by taking into account more details of the DNA assembly process? In this section, we discuss different limitations of the current version of DNA Weaver, as well as possible solutions or mitigations that will be the subject of future work.

\subsection{Assembly specificities and limitations of the graph-based approach}

Despite constant progress in molecular biology techniques, the design of DNA fragments for assembly remains challenging. The graph decomposition method used in DNA Weaver is well adapted to assembly protocols relying on homologous fragments ends, such as Gibson Assembly, but the specificities of some assembly protocols require a few amendments.

First, in homology-based assemblies, an increase in the number of fragments or inter-fragment homologies can increase assembly costs by requiring a more extensive screening of the assembly products \citep{Potapov2018, Schlichting2019}. This is not currently taken into account in DNA Weaver's price calculations, although users fearing a drop in assembly efficiency can use DNA weaver's parameters to limit the number of parts in an assembly, and use location filters to avoid designing fragments with unwanted end homologies.
DNA Weaver also implements a custom homology-avoiding algorithm for Golden Gate assemblies, whose directionality relies on 4bp homologies between successive fragments, and any end-homology between two non-successive fragments can result in invalid constructs. If the shortest-path solution presents such unwanted homologies, DNA Weaver uses a back-tracking algorithm \citep{Yen1971} to iterate over the second-shortest path, third-shortest path, etc. until reaching a solution with no undesired end homologies. While the method works, it considerably slows down the solver as the backtracking algorithm has a complexity in $O(KN^3)$, where $K$ is the number of solutions considered before reaching an acceptable solution (which may be on the order of thousands), and $N$ the number of nodes in the graph. In complex scenarios involving dozens of assembly fragments, it may be advantageous to use alternative decomposition algorithms, such as implemented in the OGAB project \citep{Tsuge2015} or the Golden Hinges library (https://github.com/Edinburgh-Genome-Foundry/GoldenHinges), which specialize in the design of sequence fragments for scarless Golden Gate assembly.

Second, oligonucleotide assembly, while very similar in principle to other homology-based assemblies, presents some unique characteristics. For instance, the single-stranded oligonucleotides assembled must alternate direct and indirect sense. DNA Weaver takes this into account by post-processing the oligo assembly plan, replacing every fragment at an even position by its reverse complement. Second, the assembly method cannot assemble an odd number of oligonucleotides. While no default mechanism to observe this constraint is currently implemented in DNA Weaver, it could be implemented in several ways: (1) by specifying that only cuts producing an even number of fragments are acceptable, and using a backtracking algorithm as described in the previous paragraph to find the best solutiidentified and assumed to be preordered on with an even number of fragments. (2) by gradually adding a penalty to each edge of the graph until the number of oligonucleotides used decreases to an even number. (3) by post-processing the assembly plan to add one final anti-sense oligonucleotide to every oligonucleotide assembly featuring an odd number of fragments. Finally, while DNA Weaver enables users to avoid fragments with given characteristics, it does not come with support for secondary structure checking, nor does it support the design of ungapped oligos. As a consequence, it may be advantageous to define an assembly station relying on specialized oligonucleotide design software \citep{Richardson2006, Rydzanicz2005}, rather than on DNA Weaver's built-in decomposition algorithm.

Finally, some assembly protocols are not currently supported, notably the iterative insertion of synthetic sequences into a scaffold chromosome, as used in the final assembly steps of chromosome assembly in the Sc2.0 project \citep{Richardson2017}, or the Ligase Cycling Reaction protocol \citep{Casini2015, Schlichting2019} in which blunt-end fragments are assembled into circular plasmids via homologies with so-called \textit{bridging oligonucleotides}. However as DNA Weaver is designed to be modular and easily extensible, users can define such new assembly methods to fit the needs of their projects.


\subsection{Modelling the logistics of DNA assembly}

While focusing on the decomposition of DNA sequences into cost-optimized fragments, DNA weaver abstracts away some contextual parameters which may influence the costs and complexity of a project, and are discussed in this section.


First, while users can model the costs and limitations of the different cloning methods based on their own experience, commercial offers must be modeled on the often scarce data communicated by vendors on their pricing, lead time, and DNA sequence constraints. Ideally, vendors would provide web application programming interfaces (APIs) indicating exact DNA prices, however (1) there is no currently no such API available to the general public, and (2) the API would need to tolerate a high volume of requests, as DNA Weaver may need to evaluate the price of thousands of sequence fragments.

Second, while users can adapt the sequences to manufacture to assembly backbones, as described in Section 2.2, the characteristics of the different backbones used are not taken into account. For instance, when protocols use antibiotic selection to separate parts from final constructs, one should make sure that successive assembly stations use different resistance markers so that the constructs of one station can be used as parts in the next. Likewise, assembly standards using fluorescent counter-selection markers, such as the Yeast Toolkit \citep{Lee2015}, are less adapted to the assembly of genetic circuits expressing the same fluorescence. In conclusion, even though DNA Weaver can greatly accelerate the design of DNA construction plans, these will still need to be carefully checked before starting production.


Third, while DNA Weaver focuses on the manufacturing of a single sequence, Synthetic Biology projects may require the simultaneous assembly of several constructs, for instance when building a library of combinatorial plasmid variants. Although there is no standard method for multi-assembly strategy optimization in DNA Weaver, one practical approach to implement sequence re-use between constructs consists in first identifying common segments between constructs and assuming that these will be pre-ordered (they are made available to the supply network via a parts library station). Assembly plans are then devised for each construct in turn, and the intermediary products of every assembly are assumed to be available for free in the next assemblies (a project-specific implementation of this strategy can be found at https://github.com/Edinburgh-Genome-Foundry/galaxy_synbiocad_dnaweaver). A related limitation is that the framework evaluates each assembly step separately, disregarding the positive volume effects in the cost of operations. In practice, pooling DNA orders or DNA assembly operations can lead to signification savings in shipping costs and reagents. However, integrating these optimizations in the design of assembly plans would require a thorough modelling of the researcher's manufacturing paralellization capabilities.

\subsection{Sequence manufacturability optimization}

A few modifications of sequence's nucleotides can have significant effects on its manufacturability, and many software tools couple the computation of an assembly plan with manufacturing-oriented modifications of the requested sequence. These modifications may include the creation or removal of restriction sites, possibly via codon juggling \citep{Richardson2006,Oberortner2017}, or the introduction of pre-determined assembly scars between standard parts \citep{Vazquez-Vilar2017}. In contrast, DNA Weaver only focuses on the manufacturing of the exact sequence requested. However, it can be coupled with a sequence optimization module, such as DNA Chisel \citep{zulkower2019dna}, to optimize the nucleotides sequence towards lower manufacturing costs, as shown in Figure 8. In this example, built-in sequence analysis methods of DNA Weaver analyze the original sequence's cost graph to pinpoint the sequence locations susceptible to drive assembly cost (in this example, the location of BsmBI sites). The sequence can then locally optimized around these locations by DNA Chisel, with respect to other design constraints (in this example, the conservation of the protein sequence), resulting in a sequence manufacturable at a much lower cost.

\section{Conclusion}

The transition from DNA design to manufacturing is often a challenging step in Synthetic Biology projects, and current software solutions for DNA assembly planning focus on specific scenarios and protocols. Here, we introduced the DNA Weaver framework and showed how its supply network modeling approach could model and resolve a variety of common DNA construction scenarios, making it a versatile and customizable tool to speed up DNA assembly in a research lab, or a DNA manufacturing facility. We also described the framework's efficient graph-based sequence decomposition methods to find cost-optimal assembly plans (or fast approximates thereof) for assembly protocols such as Gibson or Golden Gate Assembly. Finally, we discussed the limitations of the approach for other assembly methods such as oligonucleotide assembly, and how the framework's features can be completed by interfacing with other applications, such as vendors APIs, specialized cloning tools, or sequence optimization tools, which will be the object of future work.